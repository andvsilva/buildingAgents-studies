\documentclass[a4paper,12pt]{article}
\usepackage[brazil, english]{babel}
\usepackage[utf8]{inputenc}
\usepackage[T1]{fontenc}
\usepackage{geometry}
\usepackage{setspace}
\usepackage{titlesec}
\usepackage{hyperref}
\usepackage{graphicx}
\usepackage{caption}
\usepackage{subcaption}
\usepackage{fancyhdr}
\setlength{\headheight}{15pt}
\addtolength{\topmargin}{-2.5pt}
\usepackage{xcolor}
\usepackage{amsmath, amssymb, bm}
\usepackage{mathtools}
\usepackage{cancel}
\usepackage{tikz}
\usepackage{newunicodechar}
\usepackage{ragged2e}
\usepackage{setspace}
\usepackage{tikz-3dplot} % Necessário para coordenadas 3D
\usetikzlibrary{intersections}
\usepackage{siunitx}
\usetikzlibrary{3d, arrows.meta}
\usepackage{booktabs}


\usepackage{color}
\definecolor{myblue}{rgb}{.8, .8, 1}

\usepackage{amsmath}
\usepackage{empheq}

\newlength\mytemplen
\newsavebox\mytempbox

\makeatletter
\newcommand\mybluebox{%
    \@ifnextchar[%]
       {\@mybluebox}%
       {\@mybluebox[0pt]}}

\def\@mybluebox[#1]{%
    \@ifnextchar[%]
       {\@@mybluebox[#1]}%
       {\@@mybluebox[#1][0pt]}}

\def\@@mybluebox[#1][#2]#3{
    \sbox\mytempbox{#3}%
    \mytemplen\ht\mytempbox
    \advance\mytemplen #1\relax
    \ht\mytempbox\mytemplen
    \mytemplen\dp\mytempbox
    \advance\mytemplen #2\relax
    \dp\mytempbox\mytemplen
    \colorbox{myblue}{\hspace{1em}\usebox{\mytempbox}\hspace{1em}}}
\makeatother

\usepackage[most]{tcolorbox}

\newtcbox{\mymath}[1][]{%
    nobeforeafter, math upper, tcbox raise base,
    enhanced, colframe=blue!30!black,
    colback=blue!30, boxrule=1pt,
    #1}

\tcbset{
    highlight math style={
        enhanced,
        colframe=red!60!black,
        colback=yellow!50,
        arc=4pt,
        boxrule=1pt,
        drop fuzzy shadow
    }
    }

\usepackage{physics}
\usepackage{pgfplots}
\pgfplotsset{compat=1.17}

\linespread{1.5}

\definecolor{ao(english)}{rgb}{0.0, 0.5, 0.0}
\definecolor{byzantium}{rgb}{0.44, 0.16, 0.39}
\newunicodechar{∘}{\circ}

\input{commands.tex}

\hypersetup{
    colorlinks=true,% make the links colored
    linkcolor=blue
}

\usepackage{setspace}
\addbibresource{bibliography.bib}

\newcommand{\printingbibliography}{%

    \pagestyle{myheadings}
    \markright{}
    \sloppy
    \printbibliography[heading=bibintoc, % add to table of contents
                   title=Refer\^encias % Chapter name
                  ]
    \fussy%
}
\PassOptionsToPackage{table}{xcolor}

\pagestyle{fancy}
\fancyhf{}
\renewcommand{\headrulewidth}{0pt}
\fancyhead[R]{\thepage}

\geometry{a4paper,top=30mm,bottom=20mm,left=30mm,right=20mm}

\titleformat*{\section}{\bfseries\large}
\titleformat*{\subsection}{\bfseries\normalsize}

\title{ \textbf{\large Multi-agent AI system \\
Building Effective Agents \\
Personal Studies} }
\author{Andr\'e Vieira da Silva}
\date{\today}

\begin{document}

\maketitle
\noindent\rule{\linewidth}{0.4pt}\\
\justifying

\section{Introduction}

Artificial Intelligence (AI) agents represent a shift from static, rule-based software to systems capable of reasoning, acting, 
and adapting in dynamic environments. This article provides a complete introduction to AI agents, covering their core concepts, 
architectures, types, applications, and challenges \cite{anthropic2024}.

\section{What Are AI Agents? }

An \textbf{AI agent}  is a software system that can perceive an environment, make decisions, and take actions to achieve a specific goal. Unlike traditional programs, AI agents are not limited to predefined execution paths; instead, they adapt their behavior based on context, feedback, and objectives.

Modern AI agents are often powered by Large Language Models (LLMs) and extended with tools, memory, and feedback loops.

\section{Agents vs Traditional Software}

\begin{table}[h]
\centering
\begin{tabular}{lll}
\toprule
\textbf{Aspect} & \textbf{Traditional Software} & \textbf{AI Agents} \\
\midrule
Behavior & Fully predefined & Adaptive and dynamic \\
Decision-making & Rule-based & Model-driven reasoning \\
Flexibility & Low & High \\
Handling uncertainty & Poor & Strong \\
Learning from context & No & Yes \\
\bottomrule
\end{tabular}
\caption{Comparison between traditional software and AI agents}
\end{table}

Agents are particularly useful when it is not possible to anticipate all execution steps in advance.

\section{Core Components of an AI Agent}

\subsection{Goal}

Every agent operates with a clear objective, such as answering a question, analyzing data, or completing a workflow. Without a well-defined goal, effective reasoning is not possible.

\subsection{Perception}

Agents receive input from various sources, including user prompts, databases, documents, APIs, or sensor data. These inputs define the current state of the environment.

\subsection{Reasoning and Planning}

Reasoning allows the agent to determine what actions should be taken next. Planning involves decomposing a goal into smaller steps and selecting appropriate strategies and tools.

\subsection{Actions}

Agents act by executing code, calling APIs, querying databases, retrieving documents, or generating reports. Actions produce observable effects in the environment.

\subsection{Memory}

Memory enables consistency and learning. Common forms include:
\begin{itemize}
  \item Short-term memory: task context and conversation state.
  \item Long-term memory: historical interactions and learned information.
  \item Retrieval memory: external knowledge accessed via vector databases (RAG).
\end{itemize}

\subsection{Feedback and Observation}

After acting, the agent observes the outcome, evaluates success or failure, and adjusts its strategy if necessary. This feedback loop enables self-correction.

\section{The Agent Loop}

Most AI agents operate within a continuous loop:

\begin{center}
\texttt{Goal $\rightarrow$ Plan $\rightarrow$ Act $\rightarrow$ Observe $\rightarrow$ Evaluate $\rightarrow$ Adjust}
\end{center}

The loop repeats until the goal is achieved or a termination condition is met.

\section{Types of AI Agents}

\subsection{Reactive Agents}

Reactive agents respond directly to inputs without memory or planning. They are simple but limited in capability.

\subsection{Deliberative Agents}

Deliberative agents maintain internal models of the environment and can plan multiple steps ahead. Most LLM-based agents fall into this category.

\subsection{Tool-Using Agents}

These agents extend reasoning capabilities by interacting with external tools such as APIs, databases, or code execution environments.

\subsection{Learning Agents}

Learning agents improve their performance over time using feedback or reinforcement mechanisms.

\subsection{Multi-Agent Systems}

Multi-agent systems consist of multiple agents working collaboratively or competitively. Each agent typically has a specialized role, increasing scalability and robustness.

\section{Multi-Agent Systems}

In a multi-agent system, responsibilities are distributed among agents. Typical roles include:
\begin{itemize}
  \item Planner Agent: task decomposition and strategy definition.
  \item Executor Agent: performs actions and tool calls.
  \item Data Agent: handles data ingestion and transformation.
  \item Analyst Agent: performs statistical or analytical tasks.
  \item Critic Agent: validates outputs and detects errors.
\end{itemize}

This structure mirrors human teams and improves system reliability.

\section{When to Use AI Agents}

\subsection{Appropriate Use Cases}

AI agents are suitable when tasks are complex, open-ended, or require iterative reasoning and adaptation.

\subsection{When Not to Use Agents}

Agents should be avoided when workflows are simple, deterministic, or highly cost-sensitive.

\section{Applications}

AI agents are used in a wide range of domains, including data analytics, software engineering, finance, customer support, robotics, and scientific research.

\section{Challenges and Risks}

Key challenges include hallucinations, evaluation difficulty, computational cost, lack of transparency, and security risks. Effective agent systems require validation, monitoring, and clear operational constraints.

\section{Future Directions}

AI agents are evolving toward greater autonomy, improved safety, standardized evaluation, and deeper integration with real-world systems. They are expected to become a foundational component of next-generation intelligent software.

\section{Conclusion}

AI agents are not merely advanced chatbots. They are goal-driven systems that plan, act, observe, and adapt using reasoning, tools, and memory. Understanding AI agents is essential for building scalable, reliable, and intelligent AI-powered applications.

\noindent\rule{\linewidth}{0.4pt}\\

\section{Building AI Agents in Python for Investment Business Models}

%%%%%%%% Bibliography 
% Os comandos para incluir as referências bibliográficas
\printingbibliography

\end{document}
